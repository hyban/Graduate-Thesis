Diffuse optical tomography (DOT) employs near-infrared light to image the concentration of chromophores and cell organelles in tissue and thereby providing access to functional parameters that can differentiate cancerous from normal tissues. This thesis describes research at the bench and in the clinic that explores and identifies the potential of DOT breast cancer imaging. The bench and clinic instrumentation differ but share important features: they utilize a very large, spatially dense, set of source-detector pairs ($10^7$) for imaging in the parallel-plate geometry. 

The bench experiments explored three-dimensional (3D) image resolution and fidelity as a function of numerous parameters and also ascertained the effects of a chest wall phantom. The chest wall is always present but is typically ignored in breast DOT. My experiments clarified chest wall influences and developed schemes to mitigate these effects. Mostly, these schemes involved selective data exclusion, but their efficacy also depended on reconstruction approach. Reconstruction algorithms based on analytic (fast) Fourier inversion and linear algebraic techniques were explored. 

The clinical experiments centered around a DOT instrument that I designed, constructed, and have begun to test (in-vitro and in-vivo). This instrumentation offers many features new to the field. Specifically, the imager employs spatially-dense, multi-spectral, frequency-domain data; it possesses the world's largest optical source-detector density yet reported, facilitated by highly-parallel CCD-based frequency-domain imaging based on gain-modulation heterodyne detection. The instrument thus measures both phase and amplitude of the diffusive light waves. Other features include both frontal and sagittal breast imaging capabilities, ancillary cameras for measurement of breast boundary profiles, real-time data normalization, and mechanical improvements for patient comfort. The instrument design and construction is my most significant contribution, but first imaging experiments with tissue phantoms and of cancer bearing breasts were also carried out. A parallel effort with simulated data has yielded important information about new reconstruction regularization issues that arise when phase and amplitude are measured. With these gains in device implementation and DOT reconstruction, my research takes valuable steps towards bringing this novel imaging technique closer to clinical utilization.
