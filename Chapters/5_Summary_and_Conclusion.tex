\chapter{Summary and Conclusion}

\section{Summary of main results}


\subsection{Chestwall}

We have used phantom experiments to investigate systematically the
effects of the chest wall on diffusion optical tomography (DOT) of the
breast.  The results lead us to several promising conclusions.

First, we have found that, when absorption contrast is of interest,
simple continuous-wave instrumentation with linearized inversion can
suffice. This finding was obtained in spite of the presence of the
chest wall phantom in close proximity to the target, which (i) renders
the inverse problem nonlinear and (ii) differs from the background
Intralipid and the target not only in absorption but also in
scattering properties.  Generally, under the conditions stated above,
time- or frequency-resolved measurements and nonlinear
image-reconstruction methods are required. We, however, have been able
to bypass these complications by appropriately restricting the data
points used in the reconstruction. We note that in
  clinical applications of DOT, the location of the chest wall
  relative to the sources and detectors is usually known;
  therefore, the approach of this paper to data restriction can
  be applied {\em in vivo}. Work remains,however, to optimize these approaches. Devising the latter, for example, may require more sophisticated
  and/or data-driven algorithms for data rejection, as well as
  experimentation {\em in vivo}. In this paper, we have demonstrated that
  the rather severe effects of the chest wall can, in principle, be
  rectified by appropriate data restriction in conjunction with a
  linear image reconstruction. The paper shows that, by means of properly restricting the data points used in image
reconstruction, it is possible to resolve a small absorptive target in
the vicinity of a spatially and optically large inhomogeneity
and that the quality of the reconstruction is almost
  unaffected by the chest wall.

Interestingly, we have also found (see Figs.~5 and 6) that the image contrast,
when averaged over the depth of a plane-parallel sample (we refer to
this quantity as the projection), is not as sensitive to
systematic errors encountered in image reconstruction as the
individual slices drawn through the medium. Thus, under certain
conditions, the nonlinearity of the inverse problem and the presence
of a scattering contrast render our image reconstruction methods
inadequate. Reconstructed slices show severe image artifacts in this
case. The projection, however, is free from these artifacts and
displays the target clearly. This finding may be significant since a
projection of the type just discussed is similar to the usual
radiographic projection, yet it displays the contrast specific to the
near-infrared spectral range.

Finally, we have developed and verified with experimental data an algebraic 
image reconstruction method, which is well suited for the use with the data 
sets restricted by the presence of the chest wall and capable of handling 
data sets as large as $2\times 10^7$ independent measurements.

\section{Limitations}

\section{Future Directions}