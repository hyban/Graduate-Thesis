\chapter{Introduction}

\section{Diffuse Optics}
Early attempts to use light for medical tissue imaging dates back to the 1930's when transillumination images of breast were used to detect ``shadows" of tumors \cite{Cutler1931}. With tissue being highly absorptive and scattering, these crude light projections were found to be too unreliable for medical use, especially when compared to the clinical standard of x-ray. Clinical imaging with light became more feasible in the 1970's and 80's with the discovery of the so-called spectral ``window" in tissue \cite{Jobsis1977,Jo1999,Jo1999a} for near-infrared (NIR) wavelengths which eventually led to the field of NIR spectroscopy and imaging. Absorption and scattering of light in tissue was found to be minimal in this ``window", permitting light to penetrate deep into tissue in order to image major chromophores such as oxy-hemoglobin (Hb$\rm{O_2}$), deoxy-hemoglobin (Hb), lipid, water ($\rm{H_2}$O), and eventually blood flow. Formally, the tissue absorption coefficient is described as the weighted sum of oxy-hemoglobin (Hb$\rm{O_2}$), deoxy-hemoglobin (Hb), lipid, and water ($\rm{H_2}$O) for a given wavelength $\mua(\lambda)$. The scattering in the tissue is modeled with a simplfied Mie-scattering approximation to give us the reduced scattering coefficient $\musp(\lambda)$.

In media where scattering $\musp$ is much larger than absorption $\mua$, the path travelled by near-infrared photons through the medium can be described as a random walk with step size of $1/\musp$. In these cases, light transport is well approximated by a diffusion equation giving rise to the field of diffuse optics \cite{Ishimaru1978,rossum_99_1,Case1967}. Over the past several decades, the diffuse transport of photons in tissue has resulted in the development of diffuse optical spectroscopy (DOS) and diffuse optical tomography (DOT) for a variety of medical applications, including clinical applications in the breast \cite{Colak1999,Ntziachristos2000,Hawrysz2000,McBride2001,Ntziachristos2002,Durduran2002,Intes2003,Culver2003,Corlu2003,Li2003,Durduran2005,Choe2005a,Yates2005,Corlu2007,Cerussi2007,Choe2009,Durduran2010}, brain \ref{Boas2002,Culver2003a,Durduran2004,Franceschini2007}, and muscle (REF REF) as well as preclinical applications in animals \ref{Siegel2003,Culver2003a}. A key advantage of these diffuse optical techniques is that they are sensitive to functional information such as total hemoglobin concentration (THC), blood oxygen saturation (St$\rm{O_2}$), scattering ($\musp$), and blood flow, all of which have been shown to be pathologically relevant biomarkers of cancer, stroke, and other disease processes.

In this thesis, I will concentrate on the imaging of oxy- and deoxy-hemoglobin and scattering in the breast with the goal of using contrast in these physiological parameters to identify breast tumors. This work builds on extensive previous research in diffuse optical tomography in pushing to study how reconstructions can be improved for clinical use and translation of quantitative DOT instrument for clinical imaging of breast tissue.

\section{Spectroscopic monitoring and imaging of breast cancer}
For women in the U.S., breast cancer is the second most commonly diagnosed cancer (after skin cancer) and the second leading cause of death (after lung cancer) \cite{Ma2013}.  While the mortality rate of breast cancer has decreased significantly in recent years, the overall number of women still affected are large. In $2012$, the American Cancer society estimated that $39,510$ of $226,870$ new cases of breast cancer would lead to death in the U.S. \cite{Jemal2010}. The National Cancer Institute put the cost of breast cancer to the U.S. economy at $>\$8$ billion (2007) \cite{Barron2008}. 

There are currently many imaging modalities in use for the detection, diagnosis and management of breast cancer. X-ray mammography is the most prevalent and widely adopted breast cancer screening tool using ionizing photons to create high resolution projection images. Ultrasound imaging probes tissue with high frequency sound waves and use reflections to locate suspicious mass. Single Photon Emission Computed Tomography (SPECT) and Positron Emission Tomography (PET) detect gamma rays emitted from radiatively tagged proteins or sugars that is injected (or inhaled) to explore metabolic pathways in search of tumors. Magnetic Resonance Imaging (MRI) where radio waves and strong magentic field to probe tissue structure. These techniques have advantages (x-ray: high resolution, Ultrasound: low cost, PET: metabolism detection, MRI: non-ionizing) and disadvantages (x-ray: ionizing, Ultrasound: resolution, PET: ionizing, MRI: cost) that lead to arguments for which modality should be a standard in breast cancer screening and diagnosis. In reality, most modern approaches to breast cancer treatment utilize a combination of technologies in accordance with an array of treatments (neoadjuvant chemotherapy, lumpectomy, mascetomy, etc) tailored to the patient.

DOT's ability to measure endogenous tissue chromophores (Hb, HbO) and exogenous optical contrast agents (ICG) allow access to functional processes and parameters which could be used to detect breast tumors with high sensitivity and specificity \cite{Zhao2015,Choe2009}. DOT is non-invasive, uses non-ionizing radiation, and relatively low cost compared to most modalities, but suffers from issues of low resolution. Despite this apparent disadvantage, several groups have shown that changes in physiological contrasts can be used to distinguish between healthy and tumor tissue. In the context of breast imaging, DOT utilizes optical fibers placed on the surface of tissues to illuminate the breast and detect the scattered light. With multiple source-detector pairs on tissue boundaries, DOT measures the light fluence on the surface in order to computationally reconstruct the 3D distribution of $\mua$ and $\musp$ within the tissue.

DOT has successfully imaged tumors in the breast in a variety of clinical situations. For example it has been shown that DOT can differentiate of malignant and benign tumors \cite{Choe2009}. It has also been shown potential for being able to statistically separating responders from partial responders to inform chemotherapy treatments \cite{Tromberg2005,Busch2013}. 

\section{DOT instrumentation}
DOT is very flexible in its implementation, with each instance having its own set of advantages and disadvantages. Typically, DOT systems vary in a) number of wavelengths used, b) number and arrangement of source detector pairs, c) detection techniques.

Multi-spectral systems allow the imaging of multiple chromophores as \textit{a priori} spectra information can be utilized thus most DOS and DOT systems will employ two wavelengths. Source-detector arrangements are generally remission and transmission geometries. In remission geometry, the source and detector is on the same surface and allow the most flexibility in the measurement of surfaces. Most remission systems are handheld and optical properties are mapped out as the device is placed at several locations on the breast. One disadvantage is that the sensitivity function is depth dependent so there are distortions in depth. 
In the transmission geometry typical source-detector arrangements in transmission DOT include the ring~\cite{pogue_95_1} and slab~\cite{grosenick_05_1,pifferi_03_1,choe_09_1} geometries. In general transmission measurements provide more sensitivity for deep tissues since the detected light in this geometry is more likely to travel through all areas of interest.

These systems can be further divided into the type of source and detection systems which fall into three categories of continuous-wave (CW), frequency-domain (FD), and time domain (TD). CW systems are the most simple where the change in amplitude is measured to determine differences in absorption and scattering. However, in these systems, the cross-talk between absorption and scattering is severe. These cross-talks are mitigated somewhat via multi-spectral techniques. Next are Frequency-domain systems in which the laser light source is modulated (typically from the MHz up to GHz range) to detect the amplitude and phase through the tissue which allow for the separation of absorption \textit{and} scattering contrasts. The FD detection scheme can be either homodyne or heterodyne of which heterodyne is preferred because one can use filtering for amplitude detection while the phase differences are retained \cite{Nissilae2002}. Time-domain systems make up the last type of DOT systems where ultra-short laser pulses are ($\leq $ps) delivered into tissue and are temporally measured... (finish)

- Things we have done here at Penn

- Where does my system fit in?

%%%%%%%%%%%%%%%%%%%%%%%%%%
In Chapter 2, I review some theory on diffuse light propagation in tissue and select image reconstruction techniques used in my work. I will go over derivations for analytical solutions to the photon diffusion equation in frequency domain for geometries and boundary conditions commonly used in diffuse optical tomography. I will also cover linear and nonlinear methods used to solve the forward and inverse problems used in my work for image reconstruction.

In Chapter 3 describe a benchtop experiment to systematically explore diffuse optical tomography in the clinically relevant geometry of the slab with a chest wall. The effect that the chest wall, which is highly absorptive and is not in the diffusive regime, on DOT reconstruction methods had not been systematically studied. The use of high spatial density source detectors pairs to characterize the effect of the chest wall on DOT resolution is discussed. Modification to the linear reconstruction methods to mitigate the chest wall effect is also discussed.

In Chapter 4, I describe the new Gen3 breast imaging device that I have built, characterized, and deployed
into the clinical setting.  A detail schematic of the electronics and the optical switches are given and the heterodyne detection scheme is reviewed in Section \ref{sec:Gen3}. The details of the data processing and correction is summarized in Section \ref{sec:preprocess}. Data and image reconstructions from phantom experiments are shown to validate the multi-spectral and frequency-domain capabilities of the device. Finally, clinical image reconstructions from cancer patients will be shown.

In Chapter 5, I summarize my work from the preceding chapters and present current problem and suggestions for further development of our research.




