\chapter{Introduction}


Diffuse optical tomography (DOT) employs near-infrared light to probe the optical properties of highly-scattering biological tissues~\cite{arridge_99_1,boas_01_1,arridge_09_1}. DOT has shown
promise in breast cancer imaging~\cite{colak_99_1,hawrysz_00_1,mcbride_01_1,culver_03_1,intes_03_1,li_03_1,choe_05_1,yates_05_1,corlu_07_1,cerussi_07_1,choe_09_1,durduran_10_1} because optical methods are sensitive to changes in physiological parameters such as blood volume and tissue oxygenation, alterations of which are biomarkers for cancer and other disease processes.  Typical source-detector arrangements in DOT include the ring~\cite{pogue_95_1} and slab~\cite{grosenick_05_1,pifferi_03_1,choe_09_1} geometries. In the latter case, both transmission~\cite{choe_09_1} and reflection~\cite{ge_08_1} measurements have been used, though transmission measurements provide more sensitivity for deep tissues since the detected light in this geometry is more likely to travel through all areas of interest.